\section{Translation from Metamath to Python}  
\hspace{\parindent}
Our approach is based on the \texttt{mmverify.py} project \cite{mmverify}, a Metamath verifier written in Python.
Originally, \texttt{mmverify.py} was designed to read theorems symbol-by-symbol from \texttt{.mm} files, which store
formal proofs in the Metamath system, and then verify their correctness. More precisely, the following
functionalities were of interest to us:

\textbf{When parsing .mm files, occurs extraction of:}
\begin{itemize}
    \item \textit{Floating hypotheses}: variables declared with the \texttt{\$f} token;
    \item \textit{Essential hypotheses}: logical assumptions introduced using the \texttt{\$e} token;
    \item \textit{Proof bodies}: defined by \texttt{\$a} (for axiomatic statements) and
    \texttt{\$p} (for provable statements);
\end{itemize}  

\textbf{Proof Verification:}  
\texttt{mmverify.py} sequentially processes proof steps using substitutions and checks whether the final step matches the expected assertion. This ensures automatic proof validation, both in normal and compressed formats.  

Our work extends the original project by modifying it to translate formal Metamath proofs into executable Python
objects aligned with our abstractions. The key transformations include the following.

\subsection{Mapping Metamath Abstractions to Python}  

The core Metamath concepts (floating hypotheses, essential hypotheses, assertions, and proofs) remain unchanged,
following their definitions in the Metamath book \cite{metamath}.
However, syntactic possibilities of describing abstractions introduce challenges. Metamath allows names with
characters that are not valid in Python (e.g., hyphens, dots, or operator-like sequences such as \texttt{.(+)} for
summation). It is also worth noting that the names of metamafs are characterized by a partial match issue (for
example, the \texttt{idi} and \texttt{id} theorems)
To address this, we map such names to randomly generated identifiers while maintaining a mapping dictionary for reversibility.  

We define the following classes:  
\begin{itemize}
    \item A class for \textit{floating arguments}, representing variables declared in Metamath with \texttt{\$f};
    \item A class for \textit{essential arguments}, corresponding to logical hypotheses defined by \texttt{\$e};
    \item A class for \textit{assertions} (\texttt{\$p}/\texttt{\$a} in Metamath), storing the final expression that
    must be derived through proof steps;
    \item A class for \textit{proofs}, organizing proof steps where each step invokes a method performing
    substitution and hypothesis validation.
\end{itemize}  

\subsection{Example of Mapping}

We use the \texttt{metamath.exe} tool to read the \texttt{set.mm} file with the \texttt{/normal} modifier.  
The theorem \texttt{mpsylsyld} (Modus Ponens combined with double syllogism inference) is recorded in the following form:  

\begin{verbatim}
    mpsylsyld.1 $e |- ph $.
    mpsylsyld.2 $e |- ( ps -> ( ch -> th ) ) $.
    mpsylsyld.3 $e |- ( ph -> ( th -> ta ) ) $.
    $( Modus ponens combined with a double syllogism inference. $)
    mpsylsyld $p |- ( ps -> ( ch -> ta ) ) $=
    wps wph wch wth wta wph wps mpsylsyld.1 a1i
    mpsylsyld.2 mpsylsyld.3
    sylsyld $.
\end{verbatim}

The first three lines represent \textit{essential hypotheses}, followed by a comment, then the conclusion (assertion),
and finally, the proof written in reverse Polish notation (last three lines).

In our dataset, this theorem is named \textit{A0K0} (a randomly assigned identifier) and is represented as follows:

\newpage
\begin{verbatim}
from typing import TypedDict
from metamath2py.classes.apply_substitution_for_generated_files import apply_substitution

class A0K0_FloatingArgs(TypedDict):
    ph: str
    ps: str
    ch: str
    th: str
    ta: str

class A0K0_EssentialArgs(TypedDict):
    essential_1: str
    essential_2: str
    essential_3: str


class A0K0:
    """"""
    def __init__(self):
        self.essential_1 = r"""|- ph"""
        self.essential_2 = r"""|- ( ps -> ( ch -> th ) )"""
        self.essential_3 = r"""|- ( ph -> ( th -> ta ) )"""

        self.assertion = r"""|- ( ps -> ( ch -> ta ) )"""

    def call(self, floatings: A0K0_FloatingArgs, essentials: A0K0_EssentialArgs):
        essential_1_substituted = apply_substitution(self.essential_1, floatings)
        if "essential_1" not in essentials:
            raise Exception("essential_1 must be in essentials")
        if essentials["essential_1"] != essential_1_substituted:
            raise Exception(f'essentials["essential_1"] must be equal '
                            f'{essential_1_substituted} '
                            f'but was {essentials["essential_1"]}')
        <skipped lines for essential_2 and essential_3, they are the similar>
        return assertion_substituted
\end{verbatim}

The method \textit{apply\_substitution} performs the substitution of floating argument values into
the essential hypotheses of the theorem.

In the \textit{Call} method, floating arguments are substituted into the essential hypotheses of the statement,
which are declared as members of the class \textit{A0K0}.
\begin{itemize}
    \item If the essential hypotheses provided as arguments do not match the class-defined essential hypotheses, an exception is raised.
    \item If the essential hypotheses passed as arguments do not match those obtained after substitution, an exception is raised.
    \item If all checks pass, the method returns the statement with the floating arguments substituted.
\end{itemize}

The proof of a statement is also represented as a Python class that defines the steps of the proof:


\begin{verbatim}

from metamath2py.classes.A0K0 import A0K0
from metamath2py.classes.VLEL import VLEL
from metamath2py.classes.SW6P import SW6P

class A0K0_proof(A0K0):
def proof(self):
    x_1 = "wff ps"
    x_2 = "wff ph"
    x_3 = "wff ch"
    x_4 = "wff th"
    x_5 = "wff ta"
    x_6 = "wff ph"
    x_7 = "wff ps"
    x_8 = self.essential_1
    x_9 = VLEL().call(
        {
            "ph": x_6,
            "ps": x_7
        },
        {
            "essential_1": x_8
        }
    )
    x_10 = self.essential_2
    x_11 = self.essential_3
    x_12 = SW6P().call(
        {
            "ph": x_1,
            "ps": x_2,
            "ch": x_3,
            "th": x_4,
            "ta": x_5
        },
        {
            "essential_1": x_9,
            "essential_2": x_10,
            "essential_3": x_11
        }
    )

    if x_12 != self.assertion:
        raise Exception(f"x_12 was equal {x_12}, "
                        f"but expected it to be equal "
                        f"to assertion: {self.assertion}")
\end{verbatim}


In the process of proof, other statements and their corresponding method \textit{Call} can be used, for example:

\begin{verbatim}
from metamath2py.classes.VLEL import VLEL
...
x_9 = VLEL().call(
    {
        "ph": x_6,
        "ps": x_7
    },
    {
        "essential_1": x_8
    }
)
...
\end{verbatim}


It is important to note that during the proof process, declared variables are used only once. This is consistent with the practice of using reverse Polish notation in the proof of Metamath.

At the end the result value at last step is compared with the assertion of the class. If the comparison
fails, an
exception is thrown. Otherwise, the statement is considered proven.


